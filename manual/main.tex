\documentclass[a4paper]{article}

%% Language and font encodings
\usepackage[english]{babel}
\usepackage[utf8x]{inputenc}
\usepackage[T1]{fontenc}
\usepackage{courier}

%% Sets page size and margins
\usepackage[a4paper,top=2cm,bottom=2cm,left=2cm,right=2cm,marginparwidth=1.75cm]{geometry}

%% Useful packages
\usepackage{amsmath}
\usepackage{graphicx}
\usepackage[colorinlistoftodos]{todonotes}
\usepackage[colorlinks=true, allcolors=blue]{hyperref}

\title{Model dverí s prístupom na RFID kartu - užívateľský manuál}
\author{Peter Hajtol}

\begin{document}
\maketitle




\section{Zapojenie}

Zapojte priložený adaptér do systému a elektriky a počkajte pár sekúnd kým sa systém naštartuje.




\section{Používanie}

\subsection{Overovanie kariet}
Priložte kartu k čítačke. Ak je karta uložená v systéme, zabliká zelené svetlo a dvere sa otvoria. Inak zabliká červené svetlo a dvere ostanú zatvorené.

\subsection{Pridanie karty do systému}
Pre pridanie karty do systému priložte pridávaciu kartu k čítačke. Po načítaní pridávacej karty začne blikať modrá kontrolka, vyzývajúca na priloženie karty, ktorá má byť pridaná. Po prečítaní karty na pridanie sa rozsvieti zelená kontrolka na 1 sekundu, signalizujúca úspešne pridanie novej karty do systému.
\\\\
\textit{Pozn.: Pridanie karty, ktorá už je evidovaná v systéme nie je povolené. Po priložení takej karty zabliká červená kontrolka a databáza ostane v nezmenenom stave.}

\subsection{Odstránenie karty zo systému}
Pre odstránenie karty zo systému priložte mazaciu kartu k čítačke. Po jej načítaní začne blikať modrá kontrolka, vyzývajúca na priloženie karty, ktorá má byť odstránená. Po priložení karty na odstránenie sa rozsvieti zelená kontrolka na 1 sekundu, signalizujúca úspešne odstránenie karty zo systému.
\\\\
\textit{Pozn.: Odstránenie karty, ktorá predtým nebola pridaná, bude signalizované blikaním červenej kontrolky. Databáza v takomto prípade ostane v pôvodnom stave.}




\section{Konfigurácia pomocou sériovej linky}
Pripojte systém k počítaču MicroUSB káblom a spusťte na počítači softvér na komunikáciu cez sériovu linku. Funkčnosť spojenia môžete otestovať zadaním náhodného reťazca, napríklad \texttt{ahoj}. Ak spojenie funguje, systém odpovie \texttt{Unknown command, please try again.}
\\\\
Príklad testovania komunikácie (\texttt{U} - užívateľ, \texttt{S} - systém):\\
\texttt{U: ahoj}\\
\texttt{S: Unknown command, please try again.}\\

\subsection{Overenie karty}
Na overenie karty použite príkaz \texttt{CARD\_CHECK}. Na príkaz systém odpovie správou \texttt{Waiting for card to be checked...} a blikaním modrej kontrolky. Identifikátor karty na overenie zadávajte cez sériovu linku, ako 4 bajty, zapísané hexadecimálne a oddelené pomlčkou. Po odoslaní platného identifáktora odpovie systém \texttt{Card (0A-1B-2C-3D) stored.} ak je karta uložená, alebo \texttt{Card (0A-1B-2C-3D) not stored.} ak nie je uložená. V prípade že ste zadali neplatný identifikátor, systém napíše \texttt{Invalid card string, try again.}
\\\\
Príklad identifikátora karty: \texttt{0A-1B-2C-3D}.
\\\\
Príklad overenia karty \texttt{0A-1B-2C-3D} (\texttt{U} - užívateľ, \texttt{S} - systém):\\
\texttt{U: CARD\_CHECK}\\
\texttt{S: Waiting for card to be checked...}\\
\texttt{U: 0A-1B-2C-3D}\\
\texttt{S: Card (0A-1B-2C-3D) stored.}\\

\subsection{Pridanie karty do systému}
Zadajte príkaz \texttt{CARD\_ADD} a počkajte kým Vás systém vyzve na zadanie identifikátora karty. Výzva sa prejaví správou \texttt{Waiting for card to be added...} na sériovej linke a blikaním modrej kontrolky na zariadení. Identifikátor karty zadávajte cez sériovu linku, ako 4 bajty, zapísané hexadecimálne a oddelené pomlčkou. Po odoslaní identifikátora karty systém odpovie buď \texttt{Card (0A-1B-2C-3D) added.} po úspešnom pridaní karty (v zátvorke bude Vami zadaný identifikátor karty) alebo \texttt{Invalid card string, try again.} ak nebol identifikátor karty v požadovanom formáte. V prípade že ste zadali platný identifikátor karty, ktorá sa ale už nachádza v databáze, sytém odpovie správou \texttt{Card (0A-1B-2C-3D) already stored.} (s Vaším identifikátorom karty).
\\\\
Príklad identifikátora karty: \texttt{0A-1B-2C-3D}.
\\\\
Príklad pridania karty \texttt{0A-1B-2C-3D} (\texttt{U} - užívateľ, \texttt{S} - systém):\\
\texttt{U: CARD\_ADD}\\
\texttt{S: Waiting for card to be added...}\\
\texttt{U: 0A-1B-2C-3D}\\
\texttt{S: Card (0A-1B-2C-3D) added.}\\

\subsection{Odstránenie karty zo systému}
Zadajte príkaz \texttt{CARD\_DELETE} a počkajte kým Vás systém vyzve na zadanie identifikátora karty. Výzva sa prejaví správou \texttt{Waiting for card to be deleted...} na sériovej linke a blikaním modrej kontrolky na zariadení. Identifikátor karty zadávajte cez sériovu linku, ako 4 bajty, zapísané hexadecimálne a oddelené pomlčkou. Po odoslaní identifikátora karty systém odpovie buď \texttt{Card (0A-1B-2C-3D) deleted.} po úspešnom odstránení karty (v zátvorke bude Vami zadaný identifikátor karty) alebo \texttt{Invalid card string, try again.} ak nebol identifikátor karty v požadovanom formáte. V prípade že ste zadali platný identifikátor karty, ktorá sa ale nenachádza v databáze, sytém odpovie správou \texttt{Card (0A-1B-2C-3D) not stored yet.} (s Vaším identifikátorom karty).
\\\\
Príklad identifikátora karty: \texttt{0A-1B-2C-3D}.
\\\\
Príklad odstránenia karty \texttt{0A-1B-2C-3D} (\texttt{U} - užívateľ, \texttt{S} - systém):\\
\texttt{U: CARD\_DELETE}\\
\texttt{S: Waiting for card to be deleted...}\\
\texttt{U: 0A-1B-2C-3D}\\
\texttt{S: Card (0A-1B-2C-3D) deleted.}\\

\subsection{Nastavenie pridávacej karty}
Zadajte príkaz \texttt{CARD\_DEFINE\_ADDING} a počkajte kým Vás systém vyzve na zadanie identifikátora karty. Výzva sa prejaví správou \texttt{Waiting for card to be set as adding card...} na sériovej linke a blikaním modrej kontrolky na zariadení. Identifikátor karty zadávajte cez sériovu linku, ako 4 bajty, zapísané hexadecimálne a oddelené pomlčkou. Po odoslaní identifikátora karty systém odpovie buď \texttt{Card (0A-1B-2C-3D) set as adding card.} po úspešnom nastavení pridávacej karty (v zátvorke bude Vami zadaný identifikátor karty) alebo \texttt{Invalid card string, try again.} ak nebol identifikátor karty v požadovanom formáte. 
\\\\
Príklad identifikátora karty: \texttt{0A-1B-2C-3D}.
\\\\
Príklad nastavenia pridávacej karty \texttt{0A-1B-2C-3D} (\texttt{U} - užívateľ, \texttt{S} - systém):\\
\texttt{U: CARD\_DEFINE\_ADDING}\\
\texttt{S: Waiting for card to be set as adding card...}\\
\texttt{U: 0A-1B-2C-3D}\\
\texttt{S: Card (0A-1B-2C-3D) set as adding card.}\\
\\
\textit{Pozn.: Pridávaciu kartu je potrebné nastaviť po každom reštarte systému.}

\subsection{Nastavenie mazacej karty}
Zadajte príkaz \texttt{CARD\_DEFINE\_DELETING} a počkajte kým Vás systém vyzve na zadanie identifikátora karty. Výzva sa prejaví správou \texttt{Waiting for card to be set as deleting card...} na sériovej linke a blikaním modrej kontrolky na zariadení. Identifikátor karty zadávajte cez sériovu linku, ako 4 bajty, zapísané hexadecimálne a oddelené pomlčkou. Po odoslaní identifikátora karty systém odpovie buď \texttt{Card (0A-1B-2C-3D) set as deleting card.} po úspešnom nastavení mazacej karty (v zátvorke bude Vami zadaný identifikátor karty) alebo \texttt{Invalid card string, try again.} ak nebol identifikátor karty v požadovanom formáte. 
\\\\
Príklad identifikátora karty: \texttt{0A-1B-2C-3D}.
\\\\
Príklad nastavenia mazacej karty \texttt{0A-1B-2C-3D} (\texttt{U} - užívateľ, \texttt{S} - systém):\\
\texttt{U: CARD\_DEFINE\_DELETING}\\
\texttt{S: Waiting for card to be set as deleting card...}\\
\texttt{U: 0A-1B-2C-3D}\\
\texttt{S: Card (0A-1B-2C-3D) set as deleting card.}\\
\\
\textit{Pozn.: Mazaciu kartu je potrebné nastaviť po každom reštarte systému.}

\subsection{Resetovanie databázy kariet}
Reset databázy spustite príkazom \texttt{RESET\_DB} cez sériovú linku. Po úspešnom resete odpovie systém hláškou \texttt{Database has been reset.}

%You can make lists with automatic numbering \dots
%\begin{enumerate}
%\item Like this,
%\item and like this.
%\end{enumerate}
%\dots or bullet points \dots
%\begin{itemize}
%\item Like this,
%\item and like this.
%\end{itemize}

\end{document}